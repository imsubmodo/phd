
\chapter{Верификация}

Достаточно большая работа была проведена за последний год в данном направлении из-за актуальности практического применения результатов данной работы на фирме. А именно была поставлена задача верификации всех коммутаторов как разарботанных своими силами, так и покупных. Соответственно и результаты здесь более практичны, чем научны. Но без платформы для верификации невозможно было получить остальные выводы в работе.

\section{Актуальность}

Затраты (временные и ресурсные) на верификацию в процессе разработки на сегодняшний день постоянно растут относительно затрат на создание RTL-описания. Роль коммутаторов в СнК возрастает [{\color{red}более подробно в главе \ref{review}}].

Тема создания тестового окружения на языке SystemVerilog срдествами библиотеки UVM с использованием библиотеки SystemC для реализации модели проектируемого блока является актуальной и важной в современном подходе к проектированию.

\section{Научная новизна}

\section{Статьи и конференции}

Есть статья в журнале ``Вопросы радиоэлектроники'', написанная совместно с Путрей Ф.М. на эту тему.

\section{Тестовое окружение}

\subsection{Создание универсального тестового окружение}

Есть презентация на фирме на тему создания и использования универсального тестового окружения.

\subsection{Создание модели}

Модель написана на языке SystemC. Сейчас реализована простая модель. Есть необходимость ее доработки для более точного и аккуратного анализа.

\subsection{Сбор статистики}

\subsubsection{Метрики}

\subsubsection{Создание средства для визуализации}

Написано на языке программирования python с использованием библиотеки Matplotlib.

\clearpage

