
\chapter{Разработка сети-на-кристалле}

В данной главе речь пойдет о разработке сети-на-кристалле. На данный момент создана базовая реализация параметризированной сети-на-кристалле. Были проведены эксперименты касательно буферизации маршрутизаторов сетей различных размерностей. Вопрос больше проработан теоретически, чем на практике. Есть много направлений работы по конкретной реализации сети-на-кристалле. Сейчас почти завершена реализации платформы (средства генерации, тестовое окружение, модель) для проведения экспериментов.

\section{Актуальность}

Сети-на-кристалле являются наиболее эффективными в вопросах межсоединения блоков в СнК с большим количеством узлов (в зависимости от конкретного случая > 6 или > 8 узлов в сети).

\section{Научная новизна}

Существует множество зарубежных публикаций на данную тему. Не вижу на данный момент конкретной новизны в моих наработках, но тема широка и есть места, где можно доработать.

\section{Практическая значимость}

\begin{enumerate}
  \item На базе сети-на-кристалле .
\end{enumerate}

\section{Маршрут проектирования}

В маршруте проектирования использованы техники и подходы, описанные в предыдущих главах. Кратко, он состоит из следующих этапов:

\begin{enumerate}
  \item Разработка базового RTL-описания.
  \item Задание конфигурационных параметров сети.
  \item Генерация выходного RTL-описания.
  \item Верификация локальных блоков сети (маршрутизатора).
  \item Верификация сети.
  \item Синтез локальных блоков сети.
  \item Синтез сети.
  \item Анализ производительности, сбор статистики.
\end{enumerate}

\section{Архитектура}

Сеть с топологие типа двумерная решетка.

\section{Буферизация маршрутизаторов сети}

Реализована буферизация входной, внутренней и выходной частей маршрутизатора. Показаны различные зависимости от глубины буферизации для сетей разной размерности.

\section{Виртуальные каналы}

Виртуальные каналы строятся на основе буферизации маршрутизаторов сети.

\section{Арбитраж}

На данный момент реализован статический арбитраж.

\subsection{Статический}

\subsection{Динамический}

\section{Реконфигурация}

\subsection{Максимальное быстродействие / Best Effort (BE)}

\subsection{Гарантированный сервис / Guaranteed Service (GS)}

\section{Качество сервиса / Quality of Service (QoS)}

\section{Интеграция механизма синхронизация}

\clearpage

